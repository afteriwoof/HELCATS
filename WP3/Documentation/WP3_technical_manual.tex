%\documentclass[10pt,preprint]{aastex}

\documentclass[12pt, a4paper, oneside]{article}

\usepackage{helvet}
\renewcommand{\familydefault}{\sfdefault}

\usepackage[top=0cm, bottom=7cm, left=2cm, right=2cm]{geometry}

\usepackage{pdfpages}

\usepackage{multicol}


\newcommand{\Rs}{$ R_{\odot}$}
\newcommand{\pB}{$pB$}
\newcommand{\de}{$^\circ$}

\textfloatsep 8.0pt


\usepackage[small]{titlesec}

\setlength{\topmargin}{1pt}%	(gap above headers)
\setlength{\topskip}{1pt}	%	(between header and text}

\setlength{\parsep}{10pt}	%	(gap between paragraphs)
\setlength{\parindent}{20pt}	%	(indentation of paragraphs)

\setlength{\floatsep}{2pt} 	%	(space between floats (eg figures))
\setlength{\textfloatsep}{20pt}%	(space between floats and text)
\setlength{\abovecaptionskip}{2pt}	%(space above caption)
\setlength{\belowcaptionskip}{2pt}	%(space below caption)


\usepackage{subfigure}
\usepackage{natbib}
\usepackage{graphics}
\usepackage{graphicx}
\usepackage[outercaption]{sidecap}
\usepackage{footnote}

\usepackage{wrapfig}


\newenvironment{packed_item}{
\begin{itemize}
  \setlength{\itemsep}{0pt}
  \setlength{\parskip}{0pt}
  \setlength{\parsep}{0pt}
}{\end{itemize}}

\newenvironment{packed_enum}{
\begin{enumerate}
  \setlength{\itemsep}{0pt}
  \setlength{\parskip}{0pt}
  \setlength{\parsep}{0pt}
}{\end{enumerate}}

\begin{document}
%\textcolor{red}{[]}
%\tableofcontents

%\renewcommand{\caption}{\scriptsize\itshape}

%\pagenumbering{roman}
\pagenumbering{gobble}

%\setcounter{page}{-1}





\begin{center}
{\sc \Large HELCATS: WP3 Technical Manual}
\end{center}

 
 %\includegraphics[scale=0.5]{../images/ifa_logo.pdf}
\begin{center}
\begin{tabular}{ll}
\textit{Editor}: Jason P. Byrne -- Oct.~2015
\end{tabular}
\end{center}
 
%\vskip 0.05 in
%{\textsc{Summary}}

%\end{document}
\section*{\sc Task 3.1: Geometrical Modelling of STEREO/HI CMEs}

The WP3 catalogue\footnote{http://www.helcats-fp7.eu/catalogues/wp3\_cat.html} is generated by inspection and characterisation of the J-maps for the CMEs in the WP2 catalogue, by the following steps (on the {\bf stereo-ops} machine at RAL Space):

\begin{enumerate}

\item The code list.pro is run to generate a list of all fair and good events, to remove the poor events for the tracking. This code is in directory\\
/soft/ukssdc/share/Solar/HELCATS/codes/list.pro\\

\item From the list of fair and good events, a J-map for each event is called in the code\\
/soft/ukssdc/share/Solar/HELCATS/codes/jmap\_widget\_pa\_final.pro\\
In IDL the code is compiled as .r jmap\_widget\_pa\_final and then called as, e.g.,\\
jmap\_widget\_pa, `A', 2008, 02, 01, `01', /dofit, posa=80 \\
where the `01' entry corresponds to the first CME to be tracked on that day (so a small number of events are `02' if they are the second CME to be tracked on that day), the `dofit' keyword performs the model fitting to the J-map clicked tracks, and the `posa' is the position angle suggested as pa\_mid in the WP2 observational catalogue.

\item In WP3 each CME track is characterised 5 times by a point-\&-click along the bright front/ridge corresponding to the front of the CME (along the position angle chosen to generate the J-map). Two output files are produced for each track, e.g.: \\
/soft/ukssdc/share/Solar/HELCATS/tracks/HCME\_A\_\_20080201\_01\_PA080.dat \\
which contains the 5 point-\&-clicks date-time, distance (in Helioprojective-radial coordinates), J-map position angle (PA), and spacecraft (A/B); and\\
/soft/ukssdc/share/Solar/HELCATS/tracks/HCME\_A\_\_20080201\_01\_PA080.dat\_fit \\
which contains the 5 resulting fittings of each of the three methods: Fixed Phi, Self-Similar Expansion, and Harmonic Mean.

\end{enumerate}



\end{document}
