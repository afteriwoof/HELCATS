%\documentclass[10pt,preprint]{aastex}

\documentclass[12pt, a4paper, oneside]{article}

\usepackage{helvet}
\renewcommand{\familydefault}{\sfdefault}

\usepackage[top=0cm, bottom=7cm, left=2cm, right=2cm]{geometry}

\usepackage{pdfpages}

\usepackage{multicol}


\newcommand{\Rs}{$ R_{\odot}$}
\newcommand{\pB}{$pB$}
\newcommand{\de}{$^\circ$}

\textfloatsep 8.0pt


\usepackage[small]{titlesec}

\setlength{\topmargin}{1pt}%	(gap above headers)
\setlength{\topskip}{1pt}	%	(between header and text}

\setlength{\parsep}{10pt}	%	(gap between paragraphs)
\setlength{\parindent}{20pt}	%	(indentation of paragraphs)

\setlength{\floatsep}{2pt} 	%	(space between floats (eg figures))
\setlength{\textfloatsep}{20pt}%	(space between floats and text)
\setlength{\abovecaptionskip}{2pt}	%(space above caption)
\setlength{\belowcaptionskip}{2pt}	%(space below caption)


\usepackage{subfigure}
\usepackage{natbib}
\usepackage{graphics}
\usepackage{graphicx}
\usepackage[outercaption]{sidecap}
\usepackage{footnote}

\usepackage{wrapfig}


\newenvironment{packed_item}{
\begin{itemize}
  \setlength{\itemsep}{0pt}
  \setlength{\parskip}{0pt}
  \setlength{\parsep}{0pt}
}{\end{itemize}}

\newenvironment{packed_enum}{
\begin{enumerate}
  \setlength{\itemsep}{0pt}
  \setlength{\parskip}{0pt}
  \setlength{\parsep}{0pt}
}{\end{enumerate}}

\begin{document}
%\textcolor{red}{[]}
%\tableofcontents

%\renewcommand{\caption}{\scriptsize\itshape}

%\pagenumbering{roman}
\pagenumbering{gobble}

%\setcounter{page}{-1}





\begin{center}
{\sc \Large HELCATS: WP Technical Manuals Guide}
\end{center}

 
 %\includegraphics[scale=0.5]{../images/ifa_logo.pdf}
\begin{center}
\begin{tabular}{ll}
\textit{Editor}: Jason P. Byrne -- Oct.~2015
\end{tabular}
\end{center}
 
 
 \setlength{\parskip}{1em}
%\vskip 0.05 in
%{\textsc{Summary}}

%\end{document}
%\section*{\sc Task 2.1: Manual Cataloguing of STEREO/HI CMEs}

%\noindent In the context of the HELCATS project as a techniques service, it is important that the software developed to produce the various deliverables be gathered and published for transparency and knowledge sharing. The following is a guide of possible steps to achieve this, that have so far been undertaken for deliverables D2.2 and D2.3 pertaining to the HI CME Catalogues in WP2 and WP3. Please read through it and, if able, implement the steps outlined, or else talk to me about adding your contribution.

%\noindent Version control software, such as SVN and Git, are very useful for merging and archiving edits to codes and documents that are shared between contributing collaborators. The GitHub resource online provides a way to house such content for easy sharing and accessibility.

\noindent A HELCATS GitHub repository has been created at:

 https://github.com/afteriwoof/HELCATS

\noindent In it are directories for each of the WPs, with example sub-directories for Codes/ and Documentation/. In WP2/ and WP3/ the codes used to generate the HI CME catalogue are held in the Codes/ directories and details on their use is available in their respective Technical Manuals in the Documentation/ directories:

HELCATS/WP2/Documentation/WP2\_Technical\_Manual.pdf

HELCATS/WP3/Documentation/WP3\_Technical\_Manual.pdf


The WP2 observational catalogue\footnote{http://www.helcats-fp7.eu/catalogues/wp2\_cat.html} is generated via the following steps (on the STEREO-OPS machine at RAL Space with the environment variables listed in the appendix below):

\begin{enumerate}

\item Open the HI1 image file to be inspected. \\
E.g., for the date 2008-02-01 execute the command:\\
\textit{gv /data/ukssdc/STEREO/stereo\_work/jaq/CME\_LIST\_PLOTS/\\2008\_A\_DIFF/HI1A\_20080201\_diff.pdf}

\item Into the respective year file in the HELCATS directory is entered the CME date and time (of first appearance), the north and south position angles, a central position angle (deemed best for performing a J-map tracking of the event) and quality index (0, 1 or 2).\\
E.g., for the date 2008-02-01 in file \\
\textit{\$HELCATS/HI\_catalogue/STA2008.txt}\\
there is an entry for a CME with parameters\\
\textit{date 01 $|$ month 02 $|$ hour 10 $|$ min 49 $|$ pa\_N 55 $|$ pa\_mid 80 $|$ pa\_S 95 $|$ quality 1}

\item Run the code {\bf \textit{create\_wp2\_catalogue.pro}} in directory\\
	\textit{\$HELCATS/codes/} \\
	This procedure involves the following main steps.

	\begin{enumerate}
	\item Run the code {\bf \textit{combine\_wp2\_lists.pro}} to collate the yearly text files into a single text file in the appropriate format for the observational catalogue. This generates the files \textit{STEREO-$[$A$|$B$]$\_CME\_LIST\_WP2.txt}.

	\item Run the script {\bf \textit{process\_wp2\_cat.sh}} to merge the STEREO-A and -B lists into a single time ordered catalogue, remove the `Halo' field and output in ASCII, JSON and VOTable formats. The resulting files are respectively named in the convention:\\ \textit{HCME\_WP2\_Vnn.$[$txt$|$json$|$vot$]$}.

	\end{enumerate}
	
\end{enumerate}


%%%%%%%%%%%%%%%%%%
\newpage

\section*{\sc Task 3.1: Geometrical Modelling of STEREO/HI CMEs}

The WP3 catalogue\footnote{http://www.helcats-fp7.eu/catalogues/wp3\_cat.html} of CME kinematics based on geometrical modelling in the HI field-of-view is generated from an inspection and characterisation of the J-maps for the CMEs in the WP2 catalogue of CME observations, by the following steps (on the STEREO-OPS machine at RAL Space):

\begin{enumerate}

\item Run the code {\bf \textit{combine\_wp3\_lists.pro}} to generate a list of all fair and good events, i.e., ignoring the poor events for the tracking. This code resides in the \textit{codes} directory:\\
\textit{\$HELCATS/codes/}\\
An output file is produced in the \textit{WP3\_catalogue} directory for each of the two spacecraft:\\
\textit{STEREO-$[$A$|$B$]$\_CME\_TRACKING\_LIST.txt}

\item Run the code {\bf \textit{jmap\_widget\_pa\_final.pro}} on each event in the list of fair and good events to produce a J-map at the specified angle for tracking. Note, the code is compiled as .r jmap\_widget\_pa\_final and then called as, e.g.,\\
{\bf IDL$>$ jmap\_widget\_pa, `A', 2008, 02, 01, `01', /dofit, posa=80} \\
where the `01' entry corresponds to the first CME to be tracked on that day (so a small number of events are `02' if they are the second CME to be tracked on that day). The `dofit' keyword performs the model fitting to the J-map clicked tracks, and `posa' is the position angle suggested as pa\_fit in the WP2 observational catalogue.

\item In WP3 each CME track is characterised 5 times by a point-\&-click along the bright front/ridge corresponding to the front of the CME (along the position angle chosen to generate the J-map). Two output files are produced for each track and saved in the \textit{tracks} directory, e.g.: \\
\textit{\$HELCATS/tracks/HCME\_A\_\_20080201\_01\_PA080.dat} \\
which contains the 5 point-\&-clicks date-time, distance (in Helioprojective-radial coordinates), J-map position angle (PA), and spacecraft (A/B); and\\
\textit{\$HELCATS/tracks/HCME\_A\_\_20080201\_01\_PA080.dat\_fit} \\
which contains the 5 resulting fittings of each of the three methods: Fixed Phi, Self-Similar Expansion, and Harmonic Mean.

\item Run the code {\bf \textit{wp3\_single\_fits.pro}} to generate single-fits of each J-map track in addition to the 5-time average fits above, e.g., for the Ahead spacecraft:\\
{\bf IDL$>$ wp3\_single\_fits, spc=`A' [, /quiet, /test]} \\
This outputs additional files appended with \textit{\_single}, e.g.:\\
\textit{\$HELCATS/tracks/HCME\_A\_\_20080201\_01\_PA080.dat\_single}


\item Run the code {\bf \textit{create\_wp3\_catalogue.pro}} in directory\\
	\textit{\$HELCATS/codes/} \\
	This procedure involves the following main steps.

	\begin{enumerate}
	\item Run the code {\bf \textit{combine\_wp3\_tracks.pro}} to collates the yearly text files and the J-map tracks into a single text file in the appropriate format for the catalogue, i.e., containing the relevant parameters from the geometrical modelling. An output file is produced in the \textit{WP3\_catalogue} directory for each of the two spacecraft:\\
	\textit{STEREO-$[$A$|$B$]$\_CME\_LIST\_WP3.txt}.

	\item Run the script {\bf \textit{process\_wp3\_cat.sh}} to merge the STEREO-A and -B lists into a single time-ordered catalogue and output in ASCII, JSON and VOTable formats. The resulting files are respectively named in the convention:\\ \textit{HCME\_WP3\_Vnn.$[$txt$|$json$|$vot$]$}.

	\end{enumerate}



\end{enumerate}


%%%%%%%

\vspace{1cm}

\section*{\sc Appendix}

\subsubsection*{\sc Environment variables on STEREO-OPS at RAL Space:}
setenv HELCATS ``/soft/ukssdc/share/Solar/HELCATS"
\\
setenv HI\_CATALOGUE ``/soft/ukssdc/share/Solar/HELCATS/HI\_catalogue"
\\
setenv WP2\_CATALOGUE ``/soft/ukssdc/share/Solar/HELCATS/WP2\_catalogue"
\\
setenv WP3\_CATALOGUE ``/soft/ukssdc/share/Solar/HELCATS/WP3\_catalogue"
\\
setenv HELCATS\_CODES ``/soft/ukssdc/share/Solar/HELCATS/codes"
\\
setenv HI\_TRACKS ``/soft/ukssdc/share/Solar/HELCATS/tracks"





\end{document}
