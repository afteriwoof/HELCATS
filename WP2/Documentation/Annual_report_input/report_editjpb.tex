\documentclass[11pt]{report}

\usepackage[english]{babel}
\usepackage[utf8]{inputenc}
\usepackage{graphicx}
\usepackage{titlesec}
\usepackage{natbib}
\usepackage{amsmath}
\usepackage{footnote}
\usepackage{geometry}
\usepackage[document]{ragged2e}
\usepackage{times}

\setlength\textwidth{6in}
\setlength\textheight{8.25in}
\setlength\footskip{1in}
\setlength\oddsidemargin{0.15in}
\setlength\topmargin{0.5in}
\interfootnotelinepenalty=10000
\makeatletter

\renewcommand{\@makechapterhead}[1]{%
\vspace*{0 pt}%
{\setlength{\parindent}{0pt} \raggedright \normalfont
\bfseries\large
\ifnum \value{secnumdepth}>1
   \if@mainmatter\thechapter.\ \fi%
\fi
#1\par\nobreak\vspace{0 pt}}}
\makeatother

\begin{document}

\textbf{WP2 Task 2.1: Manual Cataloguing of STEREO/HI CMEs (Task Lead: STFC)}
\begin{justify}
Work Package 2 (WP2) is core to the HELCATS project, in that it provides the fundamental 
manually-generated coronal mass ejection (CME) catalogue on which the 
subsequent CME-oriented WPs, such as WP3, are based. It is generated by the visual inspection of heliospheric images from the STEREO/HI-1 
observations, from the start of the mission science phase in April 2007. Through extensive discussion 
within the HELCATS consortium, the format of the manual CME catalogue (Task 2.1 of WP2) contains the following six fields:

\begin{itemize}
\item A unique CME identifier.
\item The time of first observation of the CME in the HI-1 field of view (UTC).
\item The spacecraft (A or B).
\item The northernmost position angle extent of the CME (degrees).
\item The southernmost position angle extent of the CME (degrees).
\item A quality flag indicating whether the CME is considered \emph{poor}, \emph{fair} 
or \emph{good}.
\end{itemize}

\noindent
The unique CME identifier is a string containing both the date on which the CME is first observed
in HI-1 and the observing spacecraft, with an additional two-digit number to 
differentiate between multiple CMEs occurring on the same day. So, for example, a CME observed by STEREO-A on 31~Dec.~2007 will have the identifier ``HCME\_A\_\_20071231\_01".  CMEs that exceed the position 
angle range of the field of view are indicated with a \emph{greater than ``$>$"} or \emph{less than ``$<$"} 
symbol in the appropriate field. The quality flag has been introduced to account for the 
ambiguity that results from using human observers to identify events. This field is used as a 
means to quantify the confidence of the observer, based on the following criteria. A 
\emph{poor} event is any object spanning at least $20^\circ$ in position angle, but which 
poorly resembles a CME. A \emph{fair} event is one that resembles a CME, though not all 
observers may be convinced that this is the case (due to some limitation in the event observation such as a faint, disjoint or otherwise irregular CME structure). A \emph{good} event is one that is 
unquestionably a CME. To date, this cataloguing has been completed for the years 2007$-$2010, 2011 and 2013,  providing the observational properties of over 1000 events.
\end{justify}
\vspace*{4mm}

\newpage
\textbf{WP3 Task 3.1: Geometrical modelling of STEREO/HI CMEs (Task Lead: STFC, UNIGRAZ, UGOE)}
\begin{justify}
The STFC contribution to Task 3.1 WP3 is the derivation of kinematic properties of those 
CMEs manually identified in WP2.1. This is achieved by identifying the path of each 
CME in time/elongation plots (J-maps) and applying assumptions 
about their geometry and dynamics, summarised in (Davies et al. 2012). The CMEs are assumed to 
travel at a constant speed and to possess a cross-section corresponding to a circle undergoing 
self-similar expansion. Three different fitting methods are applied to each CME, corresponding 
to circles of half widths $\lambda=0^\circ$ (fixed phi), $\lambda=30^\circ$ (self-similar 
expansion) and $\lambda=90^\circ$ (harmonic mean). For a given CME the path of its (apparent) leading edge 
through a J-map is manually identified at the relevant position angle and each of the three fitting 
procedures are applied to determine a speed and direction. These values are, in turn, used to derive 
launch times for each event, which are then applicable to WP4.1.

The progress of this part of the geometrical modelling is that all events occurring during the 
odd-numbered years in both spacecraft have been completed. Events identified as \emph{poor} in WP2.1 
are excluded, as are those which are directed at position angles far from the ecliptic, due to 
the limited number of HI frames in which they appear. A number of small CMEs occur, which quickly become 
overwhelmed by subsequent, larger events and are also excluded. This process has been completed for 
a total of 622 events; 323 for STEREO-A and 299 for STEREO-B.


\end{justify}
%%\vspace*{4mm}



\end{document}

